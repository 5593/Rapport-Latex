\chapitre{ANALYSE ET SPÉCIFICATION DES BESOINS}

\contenue{
\addcontentsline{toc}{section}{Introduction}
\section*{Introduction}
La spécification et l'analyse des besoins sont des étapes cruciales dans la réalisation d'une application, permettant d'éviter le développement d'une application non satisfaisante.\\
Nous consacrerons ce chapitre à ces phases. Nous  identifions en premier lieu les acteurs ainsi que  les besoins fonctionnels et non fonctionnels. Ensuite, nous  présentons les diagrammes nécessaires.

\section{Analyse des besoins}
L'analyse fonctionnelle est une démarche qui consiste à rechercher et à caractériser les fonctions offertes par un produit pour satisfaire les besoins de son utilisateur.
\subsection{Identification des acteurs}
Un acteur représente l'abstraction d'un rôle joué par des entités externes (utilisateur, dispositif matériel ou autre système) qui interagissent directement avec le système.\\
Nous avons détecté l'existence de quatre acteurs qui interviennent dans les processus de notre application qui sont :
\begin{itemize}
\item[•] \textbf{L'administrateur}: Cet utilisateur a le droit de gérer les différents types des comptes ainsi que le site web.
\item[•] \textbf{Le client} : Cet utilisateur peut réserver et partager des taxis.
\item[•] \textbf{Le propriétaire du taxi}: Cet utilisateur peut gérer ses véhicules et leurs chauffeurs.
\item[•] \textbf{Le chauffeur} : Cet utilisateur peut gérer les demandes de réservation des taxis.
\end{itemize}

\subsection{Les besoins fonctionnels}

Les besoins fonctionnels sont les fonctionnalités et les actions que le système doit obligatoirement effectuer, ils sont issus du cahier des charges du projet après un consensus avec le client (entreprise d'accueil). L'application doit permettre :\\

\begin{itemize}
\item Aux éventuels utilisateurs la gestion des profils qui permet à toute personne de modifier son compte ou le désactiver, de consulter leur bonus de points et de gérer tout type de payement. Elle doit permettre aussi aux nouveaux utilisateurs de s'inscrire et  d'inviter leurs amis à rejoindre l'application. Après toute inscription et invitation, un bonus sera attribué. 
\item A l'administrateur la gestion des comptes qui permet la gestion des taxis à savoir la validation de leurs comptes, la consultation de leurs dossiers, leurs supervisions, la modification des informations relatives à chaque taxi et le blocage de son compte  en cas de franchissement du seuil toléré de réclamations. De plus, elle doit permettre la gestion des clients à savoir la supervision et le blocage du client et la consultation des informations relatives à celui-ci. L'application doit permettre aussi à l'administrateur  la gestion de la plateforme à savoir la modification, la suppression et l'ajout des articles (les nouveautés, les offres,….), l'ajout, la suppression, la modification des actions au menu et l'attribution des menus aux utilisateurs convenables, la gestion des média (les images, les fichiers pdf, les documents, les vidéo…).
\item Au propriétaire la gestion de la liste des chauffeurs en ayant la possibilité soit de bloquer celui qui atteint le seuil toléré de réclamations soit d'ajouter un nouveau chauffeur, la consultation de l'historique des trajectoires parcourus par celui-ci et la consultation des revenus approximatifs et réels. Elle doit permettre aussi la gestion des véhicules à savoir la consultation de leur liste en ayant la possibilité d'en ajouter et d'en supprimer  un et la consultation du trajet parcouru et de leur revenu.
\item Au client la réservation d'un taxi en précisant la destination tout en consultant les estimations en temps, en distance et en coût. Elle doit permettre aussi le partage des taxis avec d'autres clients.
\item Au chauffeur la gestion des demandes de réservation de son taxi par les clients soit en l'acceptant soit en l'ignorant. Selon la réponse, le statut du taxi change soit en « libre » soit en « occupé ».
\end{itemize}

\subsection{Les contraintes}
\begin{itemize}
\item L'application doit être déployée sur une architecture Web afin de permettre des accès simultanés (concurrents) à partir de sites distants.
\item L'application web s'exécute sur des machines multiplateforme et sur différent navigateurs.
\item Le web service doit permettre des accès multiples (de l'ordre de 500 utilisateurs) simultanés.
\item L'application web doit permettre la gestion des sessions crées simultanément.
\item Les applications web et mobiles doivent être multi-langues.
\end{itemize}

\section{Spécification fonctionnelle (branche fonctionnelle)}
À cette étape on spécifiera d'une manière claire et plus précise le résultat de l'analyse des besoins en adoptant le langage de modélisation unifié et plus précisément en présentant les différents Diagrammes de Cas d'Utilisation et en décrivant les différents scénarios de l'application en question.
\subsection{Diagrammes de cas d'utilisation}
\subsubsection{Diagramme de cas d'utilisation général}
La figure suivante représente le diagramme de cas d'utilisation général qui permet de donner une vision globale sur le comportement fonctionnel du système.
\begin{figure}[H]
\centering
\includegraphics[scale=0.52]{images/DiagrammesPNG/UseCases/UCGlobal.png}
\caption{Diagramme de cas d'utilisation général}
\end{figure}

\subsubsection{Raffinement de cas d'utilisation «Gérer profil»}
Pour le cas d'utilisation Gérer profil, nous présentons son diagramme de cas d'utilisation détaillé à travers la figure ci-dessous.
\begin{figure}[H]
\centering
\includegraphics[scale=0.5]{images/DiagrammesPNG/UseCases/UCGererProfil.png}
\caption{Diagramme de cas d'utilisation « Gérer profil »}
\end{figure}
Nous réalisons une description textuelle du scénario "Gérer payement" pour mieux connaître les modes de payement. Cette description est présentée par le tableau ci-dessous et montre le scénario qui doit être suivi par l'utilisateur.

\begin{table}[H]
\centering
\begin{tabular}{|l|m{12cm}|}
\hline
\textbf{Cas d'utilisation}   & \textbf{Gérer payement}                                                                                                                                                                                                                                                                                               \\ \hline
\textbf{Acteurs}             & Administrateur, Client, Propriétaire, Chauffeur                                                                                                                                                                                                                                                                       \\ \hline
\textbf{Pré-condition}       & S'authentifier                                                                                                                                                                                                                                                                                                        \\ \hline
\textbf{Scénario nominal}    & \begin{tabular}[c]{@{}l@{}}1-L'utilisateur clique sur l'onglet « Gérer payement »\\ 2-Il choisit l'onglet « Charger crédit »,\\ 3-Il sélectionne un pack de chargement disponible\\ 4-Il valide le pack\\ 5-Il sélectionne une carte de crédit disponible et introduit son code\\ 6-Le solde sera chargé\end{tabular} \\ \hline
\textbf{Scénario alternatif} & Si l'utilisateur ne possède pas une carte de crédit disponible dans laplateforme, il ajoute une.                                                                                                                                                                                                                      \\ \hline
\textbf{Post-condition}      & Compte chargé \\ \hline
\end{tabular}
\caption{Description textuelle du cas d'utilisation « Gérer  payement »}
\end{table}


\subsubsection{Raffinement de cas d'utilisation «Gérer comptes»}
Pour le cas d'utilisation Gérer comptes, nous présentons son diagramme de cas d'utilisation détaillé à travers la figure ci-dessous.
\begin{figure}[H]
\centering
\includegraphics[scale=0.55]{images/DiagrammesPNG/UseCases/UCGererCompte.png}
\caption{Diagramme de cas d'utilisation « Gérer comptes »}
\end{figure}
Nous réalisons une description textuelle du scénario "Valider compte". Cette description est présentée par le tableau ci-dessous et montre le scénario qui doit être suivi par l'administrateur.\\

\begin{table}[H]
\centering
\begin{tabular}{|l|m{12cm}|}
\hline
\textbf{Cas d'utilisation}   & \textbf{Valider compte}                                                                                                                                                                                                                                       \\ \hline
\textbf{Acteurs}             & Administrateur                                                                                                                                                                                                                                                \\ \hline
\textbf{Pré-condition}       & S'authentifier                                                                                                                                                                                                                                                \\ \hline
\textbf{Scénario nominal}    & \begin{tabular}[c]{@{}l@{}}1- L'administrateur clique sur l'onglet « Dossiers en cours »\\ 2- Il choisit un dossier parmi la liste des dossiers\\ 3-Il vérifie la disponibilité de tout papier obligatoire,\\ 4-Il clique sur « Valider compte »\end{tabular} \\ \hline
\textbf{Scénario alternatif} & si les documents sont non valides, l'administrateur refuse le compte                                                                                                                                                                                          \\ \hline
\textbf{Post-condition}      & Compte validé                                                                                                                                                                                                                                                 \\ \hline
\end{tabular}
\caption{Description textuelle du cas d'utilisation « Valider compte »}
\end{table}

\subsubsection{Raffinement de cas d'utilisation « Gérer plateforme »}
Pour le cas d'utilisation Gérer plateforme, nous présentons son diagramme de cas d'utilisation détaillé à travers la figure ci-dessous.
\begin{figure}[H]
\centering
\includegraphics[scale=0.42]{images/DiagrammesPNG/UseCases/UCGererPlateforme.png}
\caption{Diagramme de cas d'utilisation « Gérer plateforme »}
\end{figure}

Nous réalisons une description textuelle du scénario "Attribuer menu aux utilisateurs" pour mieux connaître  comment l'administrateur attribut les actions aux différents utilisateurs de l'application. Cette description est présentée par le tableau ci-dessous et montre le scénario qui doit être suivi par l'administrateur.

\begin{table}[H]
\centering
\begin{tabular}{|l|m{12cm}|}
\hline
\textbf{Cas d'utilisation}   & \textbf{Attribuer menu aux utilisateurs}                                                                                                                                                                                                                                               \\ \hline
\textbf{Acteurs}             & Administrateur                                                                                                                                                                                                                                                                         \\ \hline
\textbf{Pré-condition}       & S'authentifier                                                                                                                                                                                                                                                                         \\ \hline
\textbf{Scénario nominal}    & \begin{tabular}[c]{@{}l@{}}1- L'administrateur clique sur l'onglet menu\\ 2- L'administrateur sélectionner un menu parmi la liste des menus\\ 3- Il clique sur « Modifier »\\ 4- Il modifie le champ select avec le type d'utilisateur\\ 5- Il clique sur « Enregistrer »\end{tabular} \\ \hline
\textbf{Scénario alternatif} & Menu ne sera pas attribué                                                                                                                                                                                                                                                              \\ \hline
\textbf{Post-condition}      & Menu attribué                                                                                                                                                                                                                                                                          \\ \hline
\end{tabular}
\caption{Description textuelle du cas d'utilisation « Attribuer menu aux utilisateurs »}
\end{table}

\subsubsection{Raffinement de cas d'utilisation « Gérer chauffeurs »}
Pour le cas d'utilisation Gérer chauffeurs, nous présentons son diagramme de cas d'utilisation détaillé à travers la figure ci-dessous.
\begin{figure}[H]
\centering
\includegraphics[scale=0.5]{images/DiagrammesPNG/UseCases/UCGererChauffeur.png}
\caption{Diagramme de cas d'utilisation « Gérer chauffeurs »}
\end{figure}

Nous réalisons une description textuelle du scénario "Consulter historique des trajets". Cette description est présentée par le tableau ci-dessous et montre le scénario qui doit être suivi par le propriétaire.\\


\begin{table}[H]
\centering
\begin{tabular}{|l|l|}
\hline
\textbf{Cas d'utilisation}   & \textbf{Consulter historique des trajets}                                                                                                                                                                                                                                          \\ \hline
\textbf{Acteurs}             & Propriétaire                                                                                                                                                                                                                                                                       \\ \hline
\textbf{Pré-condition}       & S'authentifier                                                                                                                                                                                                                                                                     \\ \hline
\textbf{Scénario nominal}    & \begin{tabular}[c]{@{}l@{}}1- Le propriétaire clique sur l'onglet « Chauffeurs »\\ 2- Il sélectionne un chauffeur\\ 3- Il clique sur « Historique de trajet »\\ 4- Il remplit les données de recherche (date et heure)\\ 5- Une map sera affichée indiquant le trajet\end{tabular} \\ \hline
\textbf{Scénario alternatif} & \begin{tabular}[c]{@{}l@{}}Si le propriétaire ne remplit pas les données de recherche, le trajet\\ du jour courant sera automatiquement affiché.\end{tabular}                                                                                                                      \\ \hline
\textbf{Post-condition}      & Une map s'affiche indiquant un trajet                                                                                                                                                                                                                                              \\ \hline
\end{tabular}
\caption{Description textuelle du cas d'utilisation « Consulter historique des trajets »}
\end{table}

\subsubsection{Raffinement de cas d'utilisation « Gérer véhicules »}
Pour le cas d'utilisation Gérer véhicule, nous présentons son diagramme de cas d'utilisation détaillé à travers la figure ci-dessous.\\

\begin{figure}[H]
\centering
\includegraphics[scale=0.55]{images/DiagrammesPNG/UseCases/UCGererVehicule.png}
\caption{ Diagramme de cas d'utilisation « Gérer véhicules »}
\end{figure}

Nous réalisons une description textuelle du scénario "Ajouter véhicule". Cette description est présentée par le tableau ci-dessous et montre le scénario qui doit être suivi par le propriétaire.\\

\begin{table}[H]
\centering
\begin{tabular}{|l|m{12cm}|}
\hline
\textbf{Cas d'utilisation}   & \textbf{Ajouter véhicule}                                                                                                                                                                                                                       \\ \hline
\textbf{Acteurs}             & Propriétaire                                                                                                                                                                                                                                    \\ \hline
\textbf{Pré-condition}       & S'authentifier                                                                                                                                                                                                                                  \\ \hline
\textbf{Scénario nominal}    & \begin{tabular}[c]{@{}l@{}}1- Le propriétaire clique sur « Gérer véhicules »\\ 2- Il cliquer sur « Ajouter »\\ 3- Il remplit le formulaire\\ 4- Il ajoute les documents nécessaires pour le véhicule\\ 5- Il enregistre la demande\end{tabular} \\ \hline
\textbf{Scénario alternatif} & \begin{tabular}[c]{@{}l@{}}Si le propriétaire ne clique pas sur enregistrer la demande sera\\ annulée\end{tabular}                                                                                                                              \\ \hline
\textbf{Post-condition}      & Demande prête à être traitée par un administrateur                                                                                                                                                                                              \\ \hline
\end{tabular}
\caption{Description textuelle du cas d'utilisation « Ajouter véhicule »}
\end{table}

\subsubsection{Raffinement de cas d'utilisation « Réserver taxi »}
Pour le cas d'utilisation Réserver taxi, nous présentons son diagramme de cas d'utilisation détaillé à travers la figure ci-dessous.

\begin{figure}[H]
\centering
\includegraphics[scale=0.55]{images/DiagrammesPNG/UseCases/UCReserverTaxi.png}
\caption{Diagramme de cas d'utilisation « Réserver taxi »}
\end{figure}

Nous réalisons une description textuelle du scénario "Réserver taxi". Cette description est présentée par le tableau ci-dessous et montre le scénario qui doit être suivi par le client.\\

\begin{table}[H]
\centering
\begin{tabular}{|l|m{12cm}|}
\hline
\textbf{Cas d'utilisation}   & \textbf{Réserver taxi}                                                                                                                                                                                                                                                                                                                 \\ \hline
\textbf{Acteurs}             & Client                                                                                                                                                                                                                                                                                                                                 \\ \hline
\textbf{Pré-condition}       & S'authentifier et solde supérieur à 2 dinars                                                                                                                                                                                                                                                                                           \\ \hline
\textbf{Scénario nominal}    & \begin{tabular}[c]{@{}l@{}}1- Le client sélectionne sa destination\\ 2- Il clique sur un taxi parmi la liste sur la map\\ 3- Il valide la course,\\ 4- Il attend la validation par le chauffeur\\ 5- Si le chauffeur accepte la demande, la position du taxi sera affichée\\ sur la map et les autres taxis disparaissent\end{tabular} \\ \hline
\textbf{Scénario alternatif} & \begin{tabular}[c]{@{}l@{}}Si le chauffeur ignore la demande, le client doit sélectionner un autre\\ taxi\end{tabular}                                                                                                                                                                                                                 \\ \hline
\textbf{Post-condition}      & Taxi réservé                                                                                                                                                                                                                                                                                                                           \\ \hline
\end{tabular}
\caption{Description textuelle du cas d'utilisation « Réserver taxi »}
\end{table}

\subsubsection{Raffinement de cas d'utilisation « Partager taxi »}
Pour le cas d'utilisation Partager taxi, nous présentons son diagramme de cas d'utilisation détaillé à travers la figure ci-dessous.
\begin{figure}[H]
\centering
\includegraphics[scale=0.55]{images/DiagrammesPNG/UseCases/UCPartagerTaxi.png}
\caption{Diagramme de cas d'utilisation « Partager taxi »}
\end{figure}
Nous réalisons une description textuelle du scénario "Chercher course", Cette description est présentée par le tableau ci-dessous et montre le scénario qui doit être suivi par le client.\\

\begin{table}[H]
\centering
\begin{tabular}{|l|m{12cm}|}
\hline
\textbf{Cas d'utilisation}   & \textbf{Chercher course}                                                                                                                                                                                                                                                                                                                              \\ \hline
\textbf{Acteurs}             & Client                                                                                                                                                                                                                                                                                                                                                \\ \hline
\textbf{Pré-condition}       & S'authentifier                                                                                                                                                                                                                                                                                                                                        \\ \hline
\textbf{Scénario nominal}    & \begin{tabular}[c]{@{}l@{}}1- Le client clique sur l'onglet partager taxi\\ 2- Le client sélectionne sa destination et active le partage\\ 3- Une lise des demandes de partage sera affichée par le système selon\\ la,position actuel et la destination du client\\ 4- Le client sélectionne une demande\\ 5- Le client réserve le taxi\end{tabular} \\ \hline
\textbf{Scénario alternatif} & \begin{tabular}[c]{@{}l@{}}Si le client clique sur annuler ou sort de l'interface actuelle, la\\ demande sera automatiquement annulée\end{tabular}                                                                                                                                                                                                    \\ \hline
\textbf{Post-condition}      & Recherche effectuée                                                                                                                                                                                                                                                                                                                                   \\ \hline
\end{tabular}
\caption{Description textuelle du cas d'utilisation « Chercher course »}
\end{table}


\subsubsection{Raffinement de cas d'utilisation « Gérer demande de réservation »}
Pour le cas d'utilisation Gérer demande de réservation, nous présentons son diagramme de cas d'utilisation détaillé à travers la figure ci-dessous.\\

\begin{figure}[H]
\centering
\includegraphics[scale=0.65]{images/DiagrammesPNG/UseCases/UCGererDemandeReservation.png}
\caption{Diagramme de cas d'utilisation « Gérer demande de réservation »}
\end{figure}

Nous réalisons une description textuelle du scénario "Gérer demandes réservation". Cette description est présentée par le tableau ci-dessous et montre le scénario qui doit être suivi par le chauffeur.\\

\begin{table}[H]
\centering
\begin{tabular}{|l|m{12cm}|}
\hline
\textbf{Cas d'utilisation}   & \textbf{Gérer demandes réservation}                                                                                                                                                                                                                             \\ \hline
\textbf{Acteurs}             & Chauffeur                                                                                                                                                                                                                                                       \\ \hline
\textbf{Pré-condition}       & S'authentifier et il y a une demande par le client                                                                                                                                                                                                              \\ \hline
\textbf{Scénario nominal}    & \begin{tabular}[c]{@{}l@{}}1- L'application notifie le chauffeur de la nouvelle demande\\ 2- Le chauffeur affiche la demande avec leurs informations\\ 3- Le chauffeur clique sur « Accepter »\\ 4- la position du client sera affichée sur la map\end{tabular} \\ \hline
\textbf{Scénario alternatif} & \begin{tabular}[c]{@{}l@{}}Si le chauffeur ignore la demande, la position du client ne sera pas\\ affichée\end{tabular}                                                                                                                                         \\ \hline
\textbf{Post-condition}      & Position et destination client affichées                                                                                                                                                                                                                        \\ \hline
\end{tabular}
\caption{Description textuelle du cas d'utilisation « Gérer demandes réservation »}
\end{table}

\subsection{Diagrammes de séquences système}
\subsubsection{S'authentifier}
Nous présentons tout d'abord le diagramme de séquence système du cas d'utilisation « s'authentifier ».

\begin{figure}[H]
\centering
\includegraphics[scale=1]{images/DiagrammesPNG/sequenceSystem/DiagSeqAuthentification.png}
\caption{Diagramme de séquences système « S'authentifier »}
\end{figure}

\newpage
\subsubsection{Charger crédit}
Pour mieux expliquer le déroulement du cas d'utilisation «Charger crédit» nous présentons son diagramme de séquence système sur la figure si dessous.

\begin{figure}[H]
\centering
\includegraphics[scale=1]{images/DiagrammesPNG/sequenceSystem/DiagSeqGererPayement.png}
\caption{Diagramme de séquences système « Charger crédit »}
\end{figure}

\newpage
\subsubsection{Valider compte}
Pour mieux expliquer le déroulement du cas d'utilisation « Valider compte » nous présentons son diagramme de séquence système sur la figure ci-dessous.

\begin{figure}[H]
\centering
\includegraphics[scale=1]{images/DiagrammesPNG/sequenceSystem/DiagSeqValiderCompte.png}
\caption{Diagramme de séquences système « Valider compte »}
\end{figure}

\newpage
\subsubsection{Attribuer menu}
Pour mieux expliquer le déroulement du cas d'utilisation « Attribuer menu »,  nous présentons son diagramme de séquence système sur la figure ci-dessous.

\begin{figure}[H]
\centering
\includegraphics[scale=1]{images/DiagrammesPNG/sequenceSystem/DiagSeqAttribuerMenu.png}
\caption{Diagramme de séquences système « Attribuer menu »}
\end{figure}

\newpage
\subsubsection{Consulter historique des trajets}
Pour mieux expliquer le déroulement du cas d'utilisation « Consulter historique des trajets », nous présentons son diagramme de séquence système sur la figure ci-dessous.

\begin{figure}[H]
\centering
\includegraphics[scale=1]{images/DiagrammesPNG/sequenceSystem/DiagSeqConsulterHistoriqueTrajet.png}
\caption{Diagramme de séquences système « Consulter historique des trajets »}
\end{figure}

\newpage
\subsubsection{Ajouter véhicule}
Pour mieux expliquer le déroulement du cas d'utilisation « Ajouter véhicule », nous présentons son diagramme de séquence système sur la figure ci-dessous.

\begin{figure}[H]
\centering
\includegraphics[scale=1]{images/DiagrammesPNG/sequenceSystem/DiagSegAjouterVehicule.png}
\caption{Diagramme de séquences système « Ajouter véhicule »}
\end{figure}

\newpage
\subsubsection{Réserver taxi}
Pour mieux expliquer le déroulement du cas d'utilisation « Réserver taxi »,  nous présentons son diagramme de séquence système sur la figure ci-dessous.

\begin{figure}[H]
\centering
\includegraphics[scale=1]{images/DiagrammesPNG/sequenceSystem/DiagSeqReserverTaxi.png}
\caption{Diagramme de séquences système « Réserver taxi »}
\end{figure}

\newpage
\subsubsection{Chercher course}
Pour mieux expliquer le déroulement du cas d'utilisation « Chercher course », nous présentons son diagramme de séquence système sur la figure ci-dessous.

\begin{figure}[H]
\centering
\includegraphics[scale=1]{images/DiagrammesPNG/sequenceSystem/DiagSeqChercherCourse.png}
\caption{Diagramme de séquences système « Chercher course »}
\end{figure}

\newpage
\subsubsection{Gérer demande réservation}
Pour mieux expliquer le déroulement du cas d'utilisation « Gérer demande réservation », nous présentons son diagramme de séquence système sur la figure ci-dessous.

\begin{figure}[H]
\centering
\includegraphics[scale=1]{images/DiagrammesPNG/sequenceSystem/DiagSeqGererDemandeReservation.png}
\caption{Diagramme de séquences système « Gérer demande réservation »}
\end{figure}

\section{Spécification technique (branche technique)}
\subsection{Les besoins non fonctionnels}
Un besoin non fonctionnel est un besoin spécifiant des priorités du système, telles que les contraintes liées à l'environnement, à l'implémentation et les exigences en matière de performance, de dépendance de plateformes, de facilité de maintenance, d'extensibilité et de fiabilité.\\

Après avoir déterminé les besoins fonctionnels, nous présentons par la suite l'ensemble des contraintes à respecter pour garantir la performance du système :
\begin{itemize}
\item \textbf{Sécurité} : L'application doit être protégée contre tout accès non autorisé.
\item \textbf{Performance} : Notre application doit respecter un temps de réponse minimal.
\item \textbf{Ergonomie} : L'application doit être facile à accéder et ne demande pas un temps d'apprentissage. L'interface doit être conviviale afin d'inciter les utilisateurs à adopter cet outil.
\item \textbf{L'extensibilité} : Possibilité d'ajout ou de modification de nouvelles fonctionnalités.
\item Le développement d'une application IOS.
\end{itemize}

\subsection{Contraintes techniques}
\subsubsection{Framework Laravel}
Laravel est un Framework PHP libre de droits qui a fait son apparition en 2011. Il est peut-être jeune comparé aux autres de son genre, mais il se démarque par sa facilité, sa syntaxe élégante, et toutes sa documentation disponible à tous. De plus, Laravel utilise la toute dernière version de PHP et a fréquemment des patches de disponibles avec de nouveaux éléments et des mises à jour qui règlent les problèmes, ce qui prouve qu'il est en constante évolution et amélioration. En ce moment, il se base sur « Composer », le meilleur outil de dépendance qui gère des projets en PHP jusqu'à maintenant.\cite{laravel}

\subsubsection{Plateforme Android}
Android est un système d'exploitation fondé sur un noyau Linux, en open source, pour Smartphones, PDA (Personal Digital Assistant) et terminaux mobiles. Il comporte une interface spécifique, développée en java. On le trouve aussi sur d'autres appareils comme les téléviseurs et les tablettes.\\
Le logiciel est proposé de façon gratuite et librement modifiable aux fabricants de téléphones mobiles, ce qui facilite son adoption.\cite{android}

\subsubsection{Web Service}
Un Web Service est un ensemble de protocoles permettant à des applications de communiquer entre elles, et ce indépendamment de la structure et du langage employés. Il propose aux utilisateurs du web des fonctionnalités pratiques grâce à un protocole Web standard dans la plupart des cas, le protocole utilisé est SOAP. Les Web Services présentent un avantage certain notamment pour diversifier la façon de servir des données. Les styles d'architectures orientés services les plus connus sur le marché sont les architectures basées sur le protocole SOAP et l'architecture REST, nous les présentons dans les parties suivantes.

\paragraph{SOAP}
SOAP (Simple Object Access Protocol)  est une recommandation du W3C. Il s'agit du type de Web Service le plus courant, permettant de communiquer et d'échanger des messages.
Il bénéficie, de plus, d'une spécification complète et détaillée dans les normes éditées par le W3C.
SOAP autorise des objets à faire des appels à des méthodes issues d'autres objets la plupart du temps physiquement séparés (serveurs distants). Ce protocole utilise le plus souvent http comme protocole de communication, mais il peut aussi utiliser d'autres protocoles comme
SMTP. SOAP permet d'assurer la communication et la transmission de messages entre deux objets distants en autorisant l'appel à des méthodes servies par un Web Service. Lors d'un échange SOAP, le message transmis se décompose en 2 parties :\\
\begin{itemize}
\item  L'enveloppe HTTP, qui contient toutes les informations relatives au contenu du message.
\item Le corps SOAP-XML, qui est formé de données structurées (méthodes formatées, interrogation structurée d'après les besoins du serveur). Les messages SOAP reposent sur les technologies XML, ce qui permet une interopérabilité totale entre systèmes.
\end{itemize}

La figure ci-dessous explique le fonctionnement du protocole SOAP.
\begin{figure}[H]
\centering
\includegraphics[scale=0.6]{images/SOAP.png}
\caption{Protocole SOAP \cite{soapg}}
\end{figure}

Le fonctionnement de SAOP est facile à comprendre : un message SOAP est envoyé d'un expéditeur (client) à un destinataire (serveur), directement ou via des intermédiaires. Pour se faire, le client ouvre une connexion HTTP et envoie une requête SOAP qui est un document XML décrivant une méthode à invoquer sur une machine distante ainsi que ses paramètres.
Le serveur va récupérer la requête, exécuter la méthode avec les paramètres et renvoyer une réponse SOAP (document XML) au client.\cite{soap}


\paragraph{REST}
\begin{figure}[H]
\centering
\includegraphics[scale=0.4]{images/REST.png}
\caption{Web service REST\cite{rest}}
\end{figure}
REST (Representational State Transfer) est un style d'architecture pour les systèmes distribués. Le principe du REST est d'utiliser HTTP pour l'implémentation d'un Web Service, non seulement comme simple protocole de transport mais également pour définir l'API de chaque service, c'est-à-dire la définition même des messages entre client et serveur. REST se base sur le concept de ressources. L'accès à des ressources se fait à travers une URI. La communication entre le client et le serveur se fait en échangeant une représentation de cette ressource, ayant un format JSON, XML, Vidéo, image, etc. Les différentes actions possibles sur ces ressources sont données par les différents types de requête HTTP, principalement :\cite{rest}
\begin{itemize}
\item GET : Obtient une représentation d'une entrée existante.
\item PUT : Modifie une entrée existante.
\item POST : Crée une nouvelle entrée.
\item DELETE : Supprime une entrée.
\end{itemize}


\paragraph{Justification du choix du web service REST}
Les architectures de type REST deviennent, de plus en plus, des standards dans les entreprises aujourd'hui. La plupart des consultants en informatique de gestion vous le diront. Cependant, au-delà de l'effet de mode, la mise en œuvre de services Web basée sur cette architecture a de quoi séduire. Les principales motivations pour le choix de cette architecture sont:

\begin{itemize}
\item L'indépendance vis à vis du langage de programmation,
\item L'indépendance vis à vis de la plateforme sur laquelle ils sont déployés,
\item La plus grande simplicité d'implémentation car la couche transport n'a pas besoin d'être redéfinie et il n'est plus nécessaire de créer un dictionnaire de données comme on le fait avec SOAP.
\item La non intégration du format d'échange au sein des messages. Cela induit une verbosité beaucoup plus faible et donc une performance accrue et une plus grande souplesse dans l'implémentation,
\item L'utilisation de multiples formats pour l'échange de données (XML, JSON, HTML),
\item Qu'elle est plus proche de la conception et de la philosophie initiale du Web (URI, GET, POST, PUT et DELETE),
\item Pas de gestion d'états du client sur le serveur.
\end{itemize}

\section{Diagramme de déploiement}
Le diagramme de déploiement sert à représenter l'utilisation de l'infrastructure physique par le système et la manière avec laquelle les composants du système sont répartis ainsi que les relations entre eux. Les éléments utilisés par un diagramme de déploiement sont principalement les nœuds, les composants, les associations et les artefacts. Les caractéristiques des ressources matérielles et des supports de communication peuvent être précisées stéréotype.\\

La figure ci-dessous montre le diagramme de déploiement de notre application.

\begin{figure}[H]
\centering
\includegraphics[scale=0.4]{images/DiagrammesPNG/Diagrammededeploiement.png}
\caption{Diagramme de déploiement}
\end{figure}

La modélisation du diagramme de déploiement ci-dessus montre cinq nœuds :
\begin{itemize}
\item[•]  \textbf{Serveur web} : Composé d'un module de scripts PHP et contient le système de gestion de base de données MySQL.
\item[•] \textbf{Serveur google} : est un service de cartographie en ligne.
\item[•] \textbf{Poste} : Représente le navigateur utilisé pour visualiser les interfaces web.
\item[•] \textbf{Client mobile} : représente l'application mobile client.
\item[•] \textbf{Client mobile chauffeur}: représente l'application mobile chauffeur.
\item[•] \textbf{Circuit électronique} : représente le circuit électronique qui communique avec l'application mobile chauffeur pour modifier le statut du taxi.
\end{itemize}


\addcontentsline{toc}{section}{Conclusion}
\section*{Conclusion}
Dans ce chapitre nous avons spécifié les différents besoins fonctionnels et non fonctionnels des utilisateurs de l'application. Nous avons traité les différents cas d'utilisations que nous avons détaillés textuellement et graphiquement grâce aux diagrammes de séquence système modélisant les différentes interactions entre le système et les acteurs dans un ordre chronologique. Après la phase de spécification, nous passons à la phase de conception de l'application dans le chapitre suivant.









}
