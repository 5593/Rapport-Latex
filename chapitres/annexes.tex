\cover{ANNEXES}




\contenue{

%|----------------------------
%|	ANNEXE A : Questionaire
%|----------------------------

\addcontentsline{toc}{section}{Annexe A : Questionnaire}
\section*{Annexe A : Questionnaire}

%\addcontentsline{toc}{subsection}{Questions chauffeurs}
\subsection*{Questions chauffeurs :}

\begin{minipage}[c]{.7\linewidth}
\begin{figure}[H]
\centering
\includegraphics[scale=0.7]{images/Statistiques/taxi1.png}
\caption{Histogramme de nombre des propriétaires}
\end{figure}

 \end{minipage}\hfill
\begin{minipage}[c]{.3\linewidth}
40\% des chauffeurs sont les propriétaires du taxi.
\end{minipage}


\begin{minipage}[c]{.65\linewidth}
\begin{figure}[H]
\centering
\includegraphics[scale=0.8]{images/Statistiques/taxi2.png}
\caption{Graphique de nombre de chauffeurs par taxi}
\end{figure}

 \end{minipage}\hfill
\begin{minipage}[c]{.35\linewidth}
\begin{itemize}
\item 53\% des taxis ont deux chauffeurs.

\item 42\% des taxis ont un seul chauffeur.

\item 5\% des taxis ont 3 chauffeurs.
\end{itemize}
\end{minipage}



\begin{minipage}[c]{.65\linewidth}
\begin{figure}[H]
\centering
\includegraphics[scale=0.7]{images/Statistiques/taxi3.png}
\caption{Histogramme de nombre de propriétaires voulant contrôler leur taxi}
\end{figure}

 \end{minipage}\hfill
\begin{minipage}[c]{.35\linewidth}
96\% des propriétaires veulent contrôler leurs taxis à distance.
\end{minipage}



\begin{minipage}[c]{.6\linewidth}
\begin{figure}[H]
\centering
\includegraphics[scale=0.7]{images/Statistiques/taxi4.png}
\caption{Histogramme de taux d'utilisateurs}
\end{figure}

 \end{minipage}\hfill
\begin{minipage}[c]{.4\linewidth}
72\%  des chauffeurs veulent utiliser notre future application
\end{minipage}


\begin{minipage}[c]{.6\linewidth}
\begin{figure}[H]
\centering
\includegraphics[scale=0.82]{images/Statistiques/taxi5.png}
\caption{Graphique de nombre de chauffeurs par catégorie d'âge}
\end{figure}
 \end{minipage}\hfill
\begin{minipage}[c]{.4\linewidth}
\begin{itemize}
\item 40\% des chauffeurs sont âgés de 30 à 40 ans.

\item 33\% des chauffeurs sont âgés de 20 à 30 ans.

\item 27\% des chauffeurs sont âgés de plus de 40 ans.
\end{itemize}
\end{minipage}



\begin{minipage}[c]{.6\linewidth}
\begin{figure}[H]
\centering
\includegraphics[scale=0.82]{images/Statistiques/taxi6.png}
\caption{Graphique de pourcentage de chauffeurs par mode de payement de réservation}
\end{figure}
 \end{minipage}\hfill
\begin{minipage}[c]{.4\linewidth}
\begin{itemize}
\item 85\% des chauffeurs veulent que le client paye directement.

\item  14\% veulent que les clients achètent un pack forfaitaire.
\end{itemize}
\end{minipage}



\subsection*{Questions clients :}

\begin{minipage}[c]{.6\linewidth}
\begin{figure}[H]
\centering
\includegraphics[scale=0.85]{images/Statistiques/Client1.png}
\caption{Graphique de taux d'utilisateurs}
\end{figure}
 \end{minipage}\hfill
\begin{minipage}[c]{.4\linewidth}
\begin{itemize}
\item 70\% des clients veulent utiliser notre 
future application.

\item 15\% ne veulent pas.

\item 15\% n’ont pas de réponse.
\end{itemize}
\end{minipage}


\begin{minipage}[c]{.6\linewidth}
\begin{figure}[H]
\centering
\includegraphics[scale=0.85]{images/Statistiques/Client2.png}
\caption{Histogramme  de nombre de clients par catégorie d'âge}
\end{figure}
 \end{minipage}\hfill
\begin{minipage}[c]{.4\linewidth}
\begin{itemize}
\item 40\% des clients sont âgés de 20 à 30 ans.

\item  26\% des clients sont âgés de 30 à 40 ans.

\item  19\% des clients sont âgés de plus de 40 ans.

\item  15\% des clients sont âgés de 16 à 20 ans.
\end{itemize}
\end{minipage}

\begin{minipage}[c]{.6\linewidth}
\begin{figure}[H]
\centering
\includegraphics[scale=0.7]{images/Statistiques/Client3.png}
\caption{Histogramme de pourcentage de clients par mode de payement de réservation}
\end{figure}
 \end{minipage}\hfill
\begin{minipage}[c]{.4\linewidth}
\begin{itemize}
\item 60\% des clients veulent acheter un Pack forfaitaire.

\item 40\% veulent payer par course.
\end{itemize}
\end{minipage}






%|----------------------------
%|	Annexe B : Maquettes IHM
%|----------------------------
\newpage
\addcontentsline{toc}{section}{Annexe B : Maquettes IHM}
\section*{Annexe B : Maquettes IHM}
Les maquettes simulent de manière réaliste le fonctionnement de la future interface de l’application.

\subsection*{Application mobile côté client}
Tout d’abord, le client doit créer un compte. Ensuite, il se connecte pour qu’il puisse accéder aux fonctionnalités de l’application.


\begin{minipage}[c]{.46\linewidth}
\begin{figure}[H]
\centering
\includegraphics[scale=0.6]{images/maquettes-IHM/connexion.png}
\caption{IHM - Connexion}
\end{figure}

 \end{minipage}\hfill
\begin{minipage}[c]{.46\linewidth}
\begin{figure}[H]
\centering
\includegraphics[scale=0.6]{images/maquettes-IHM/inscription.png}
\caption{IHM - Création compte}
\end{figure}
\end{minipage}

\vspace*{2cm}
Dès que le client se connecte, il peut voir la localisation des taxis par rapport à sa position. Il choisit le taxi le plus proche. Suite au clique sur le taxi, les informations, telles que le temps approximatif pour que le taxi arrive et sa capacité, s’affichent. Si le client décide de réserver le taxi, il clique sur commander pour afficher les informations concernant le chauffeur puis il clique sur contacter pour informer le chauffeur de la réservation et de l’emplacement de ce dernier.



\begin{minipage}[c]{.46\linewidth}
\begin{figure}[H]
\centering
\includegraphics[scale=0.6]{images/maquettes-IHM/homeConnected.png}
\caption{IHM - Position taxis}
\end{figure}

 \end{minipage}\hfill
\begin{minipage}[c]{.46\linewidth}
\begin{figure}[H]
\centering
\includegraphics[scale=0.6]{images/maquettes-IHM/taxiCliked.png}
\caption{IHM - Information taxi sélectionné}
\end{figure}

\end{minipage}



\vspace*{1cm}
\begin{minipage}[c]{.46\linewidth}
\begin{figure}[H]
\centering
\includegraphics[scale=0.6]{images/maquettes-IHM/taxiCommandeAlerte.png}
\caption{IHM - Information chauffeur et confirmation réservation}
\end{figure}

 \end{minipage}\hfill
\begin{minipage}[c]{.46\linewidth}
\begin{figure}[H]
\centering
\includegraphics[scale=0.6]{images/maquettes-IHM/taxiEnRoute.png}
\caption{IHM - Position du taxi par rapport au client}
\end{figure}

\end{minipage}

\newpage
Le client peut aussi accéder au menu pour choisir une action. Il peut consulter son espace personnel et modifier ses informations. Il peut aussi inviter des amis à rejoindre l’application.



\begin{minipage}[c]{.46\linewidth}
\begin{figure}[H]
\centering
\includegraphics[scale=0.6]{images/maquettes-IHM/menu.png}
\caption{IHM - Menu}
\end{figure}

 \end{minipage}\hfill
\begin{minipage}[c]{.46\linewidth}
\begin{figure}[H]
\centering
\includegraphics[scale=0.6]{images/maquettes-IHM/espacePersonnel.png}
\caption{IHM - Espace personnel}
\end{figure}

\end{minipage}


\begin{minipage}[c]{.46\linewidth}
\begin{figure}[H]
\centering
\includegraphics[scale=0.6]{images/maquettes-IHM/inviterAmis.png}
\caption{IHM - Inviter amis}
\end{figure}

 \end{minipage}\hfill
\begin{minipage}[c]{.46\linewidth}
\begin{figure}[H]
\centering
\includegraphics[scale=0.6]{images/maquettes-IHM/invitAmiSended.png}
\caption{IHM - Invitation envoyée}
\end{figure}

\end{minipage}





\subsection*{Application mobile côté chauffeur}
Tout d’abord, le chauffeur doit se connecter pour pouvoir accéder à son compte.\\
Lorsqu’une demande de réservation d’un client arrive, le chauffeur peut ou bien accepter, ou bien refuser la demande. Il peut aussi changer le statut de son taxi soit en libre soit en occupé.


\begin{minipage}[c]{.46\linewidth}
\begin{figure}[H]
\centering
\includegraphics[scale=0.6]{images/maquettes-IHM/AccepterReservationClient.png}
\caption{IHM - Confirmation réservation du client}
\end{figure}

 \end{minipage}\hfill
\begin{minipage}[c]{.46\linewidth}
\begin{figure}[H]
\centering
\includegraphics[scale=0.6]{images/maquettes-IHM/ChangerStatutChauffeur.png}
\caption{IHM - Statut taxi}
\end{figure}

\end{minipage}



\newpage
\subsection*{Application web}
L’application web a deux  types de maquettes :
\begin{itemize}
\item[•] La première présente le tableau de bord qui sera utilisé par le propriétaire, l’administrateur et client.
\item[•] La deuxième présente le site vitrine.
\end{itemize}

\begin{figure}[H]
\centering
\includegraphics[scale=0.45]{images/maquettes-IHM/web/dasboard-IHM.png}
\caption{IHM - Tableau de bord}
\end{figure}

\begin{figure}[H]
\centering
\includegraphics[scale=0.45]{images/maquettes-IHM/web/web-IHM.png}
\caption{IHM - Site vitrine}
\end{figure}




%|----------------------------
%|	Annexe C : Interfaces Web
%|----------------------------

\addcontentsline{toc}{section}{Annexe C : Interfaces Web}
\section*{Annexe C : Interfaces Web}


\begin{minipage}[c]{.35\linewidth}
Notre site web peut être consulté via un smartphone et une tablette grâce à son design responsive.
 \end{minipage}\hfill
\begin{minipage}[c]{.6\linewidth}
\begin{figure}[H]
\centering
\includegraphics[scale=0.14]{images/imprimeEcran/siteWeb/responsive.png}
\caption{Site web responsive}
\end{figure}
\end{minipage}

\vspace*{0.7cm}
L’utilisateur peut s’inscrire puis se connecter à notre application à travers les interfaces ci-dessous.

\begin{figure}[H]
\centering
\includegraphics[scale=0.3]{images/imprimeEcran/siteWeb/login.png}
\caption{Page de connexion et d'inscription}
\end{figure}




%|----------------------------
%|	Annexe D : Interfaces application mobile
%|----------------------------

\addcontentsline{toc}{section}{Annexe D : Interfaces application mobile}
\section*{Annexe D : Interfaces application mobile}


\begin{minipage}[c]{.35\linewidth}
Si une erreur inattendue survient, l’interface ci-dessous s’affiche au lieu de l’erreur par défaut d’Android.
 \end{minipage}\hfill
\begin{minipage}[c]{.6\linewidth}
\begin{figure}[H]
\centering
\includegraphics[scale=0.14]{images/imprimeEcran/android/erreur.png}
\caption{Page d'erreur - Application Mobile}
\end{figure}

\end{minipage}


\vspace{1cm}
\begin{minipage}[c]{.35\linewidth}
Le client peut accéder au menu suivant pour faire les fonctionnalités telles que : la gestion de son compte, le partage du taxi ...
 \end{minipage}\hfill
\begin{minipage}[c]{.6\linewidth}
\begin{figure}[H]
\centering
\includegraphics[scale=0.14]{images/imprimeEcran/android/menu.png}
\caption{Menu client - Application mobile}
\end{figure}

\end{minipage}



\newpage
Les interfaces ci-dessous s’affichent lorsque la connexion internet et le GPS sont désactivés et demandent aux utilisateurs de les activés avant d’utiliser l’application.\\


\begin{minipage}[c]{.46\linewidth}
\begin{figure}[H]
\centering
\includegraphics[scale=0.13]{images/imprimeEcran/android/gpsDisp.png}
\caption{Alerte service de localisation - Application Mobile}
\end{figure}

 \end{minipage}\hfill
\begin{minipage}[c]{.46\linewidth}
\begin{figure}[H]
\centering
\includegraphics[scale=0.13]{images/imprimeEcran/android/internetDis.png}
\caption{Alerte Connexion internet - Application Mobile}
\end{figure}

\end{minipage}


















}