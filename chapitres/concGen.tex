\addcontentsline{toc}{chapter}{CONCLUSION GÉNÉRALE}
\chapter*{CONCLUSION GÉNÉRALE}
\lhead{\footnotesize CONCLUSION GÉNÉRALE}
\contenue{
\vspace*{-5mm}
Ce projet s'inscrit dans le cadre du projet de fin d'études. Le but de ce projet est de concevoir et d'implémenter une plateforme, un web service et deux applications mobiles au sein de la société SIG solution.\\

Pour se faire, nous avons mis le projet dans son contexte. Par la suite nous avons tiré les besoins fonctionnels et non fonctionnels majeurs de notre application. Ensuite, nous avons entamé la conception détaillée tout en présentant les différents diagrammes UML qui nous ont permis de mettre en œuvre notre application. Enfin, dans une quatrième partie consacrée à la réalisation, nous avons pu choisir l’environnement logiciel nécessaire à la mise en œuvre des fonctionnalités sollicitées de notre application pour pouvoir finir par présenter quelques interfaces La réalisation de ce travail était une expérience humaine et professionnelle enrichissante. La réussite d’un projet est fonction de plusieurs paramètres qui doivent être travaillés minutieusement afin de garantir le résultat attendu.\\

Ce stage de fin d’études nous a permis de mettre en œuvre nos savoirs et nos connaissances théoriques, et de les enrichir par la pratique, ainsi que de maitriser un ensemble de nouvelles technologies. Cette expérience a été une opportunité pour apprendre en avantage sur les web services, de s’approfondir dans le développement web et mobile dans sa globalité, assimiler ses concepts et manipuler ses outils.\\

Nous avons également appris à être encore plus autonomes et responsables afin de bien s’intégrer dans le milieu professionnel.\\
Ce projet nous a été une source de bénéfices valorisants tant au niveau technique qu’au niveau professionnel et relationnel.\\

Nous tenons à souligner que la solution développée reste extensible à tout type d’amélioration et d’ajout de nouveaux modules tels que :

\begin{itemize}
\item ‌La création d’une application mobile pour IOS.
\item ‌‌La sécurisation des requêtes transmises entre notre API et les autres applications.
\item ‌‌L’amélioration de la partie "Partager Taxi" afin de permettre aux amis des réseaux sociaux (facebook, twitter...) les plus proches et ayant la même destination pour partager la même course.
\item ‌‌Faciliter le mode de payement international (visa, masterCard, payPal...) et ce pour faciliter la tâche aux touristes.
\item ‌‌Déterminer avec plus de précision les valeurs approximatives d'une course telles que : le cout, la distance...
\item ‌‌La construction d’un circuit intégré et implémentation de la partie qui communique avec ce dernier. Cette partie permet le changement automatique du statut du taxi (occupée, libre).
\item ‌‌La Création d’un système de gestion de trafic base sur d'une part notre application et d'autre part les services google (wayz) pour choisir le meilleur itinéraire de point de vue trajet et encombrement (plus court, moins encombré).\\
\end{itemize}

En guise de conclusion, on dirait que notre projet se suite dans une voie en pleine expansion puisque le domaine du transport est en évolution croissante.




}
