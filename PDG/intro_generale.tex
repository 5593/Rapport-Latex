%ajouter introduction generale au table de matiers
\addcontentsline{toc}{chapter}{INTRODUCTION GÉNÉRALE}
\textbf{\Large INTRODUCTION GÉNÉRALE}\\
%\chapter*{INTRODUCTION GÉNÉRALE}
\contenue{
\lettrine[lines=2]{D}e nos jours, vu l'accélération du rythme de la vie urbaine, la surpopulation des villes, l'encombrement des routes et le problème de l'embouteillage, le secteur du transport en général et du transport en commun (collectif) rencontre beaucoup de difficultés qui rendent difficile et pénible le déplacement des citadins.\\

Ces problèmes ont leurs répercussions sur le service rendu par les taxis. En effet, ce type de transport « le Taxi » souffre de trois types majeurs de difficultés se rapportant :\\
D'une part, au client telles que la difficulté de trouver un taxi hors des rues principales et la pénurie des taxis surtout aux heures de pointes étant données que tous les gens se rendent à leur travail au même moment.
D'autre part, au taxi. En effet, la recherche des clients a deux effets négatifs, la perte du temps et d'énergie et par conséquent la pollution de l'environnement.
Et enfin, ces problèmes se rapportent au propriétaire du taxi. Celui-ci ne peut contrôler ni l'emplacement de son véhicule ni ses revenus.\\

Pour résoudre ces problèmes, il faut mettre un système dont le but est de faciliter la tâche du client de trouver un taxi sans gaspiller du temps, de réduire le coût d'une course en permettant à plusieurs clients de partager le même taxi, de faciliter le mode de paiement en permettant de régler les frais d'une course par espèce ou par carte bancaire... Pour le propriétaire du taxi, il pourra connaître en temps réel le parcours de son véhicule, recevoir directement les frais du service rendu au client et connaître ses revenus exacts.\\
 
C'est dans ce cadre que s'inscrit notre projet de fin d'études intitulé conception et développement d'un système de service de transport intelligent que nous réalisons au sein de l'entreprise  SIG SOLUTION. Il consiste à concevoir et à réaliser une application comprenant : \\

\begin{itemize}
\item Un site web pour administrer le système du service de transport.
\item Une application mobile pour les chauffeurs.
\item Une application mobile  pour les clients.
\item Un web service pour communiquer entre les différentes plateformes.\\
\end{itemize}

 
Par le biais de ce rapport, nous détaillons les différentes phases de réalisation de ce projet. Ce présent rapport s'articule autour de cinq chapitres :\\

Le premier chapitre est le cadre général du projet  dans lequel nous effectuons une présentation générale de l'organisme d'accueil, ainsi qu'une étude et critique de l'existant et détaillons la méthodologie suivie pour la réalisation de ce travail.\\

Le deuxième chapitre aura comme but l'analyse et la spécification complète des différentes fonctionnalités offertes par notre système. \\

Ensuite, le troisième chapitre sera dédié à la conception de l'application en instanciant les spécifications présentées dans le chapitre précédent et tout en se basant sur une modélisation utilisant le langage UML.\\

Enfin, le quatrième chapitre couronne le travail réalisé en exposant l'environnement matériel et logiciel utilisé et les interfaces graphiques de l'application.
 Nous finirons le rapport par une conclusion générale.
}
\newpage